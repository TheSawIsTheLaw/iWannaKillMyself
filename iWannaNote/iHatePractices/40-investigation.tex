\section{План исследований}
\subsection{Условия исследований}
Исследование проводилось на персональном компьютере со следующими характеристиками:

\begin{itemize}
\item процессор Apple M1 Pro,
\item операционная система macOS Ventura 13.5.2 (22G91),
\item 32 Гб оперативной памяти.
\end{itemize}

Для определения гиперпараметров каждой модели применялся метод поиска по сетке с опорой на значение F1-меры. Разбиение данных на выборки производилось на 4 части, 1 из которых используется в качестве тестовой, результаты приводятся для каждого разбиения.

\subsection{Исследование применимости моделей машинного обучения в методе распознавания суицидальных паттернов поведения человека по текстовым сообщениям}

\subsubsection{Градиентный бустинг}

В данном подразделе представлены параметры модели, матрицы ошибок и оценки классификатора для задействованных методов векторизации.


\subsubsection{Метод случайного леса }

В данном подразделе представлены параметры модели, матрицы ошибок и оценки классификатора для задействованных методов векторизации.


\subsubsection{Метод опорных векторов }

В данном подразделе представлены параметры модели, матрицы ошибок и оценки классификатора для задействованных методов векторизации.


\subsubsection{Метод K-ближайших соседей }

В данном подразделе представлены параметры модели, матрицы ошибок и оценки классификатора для задействованных методов векторизации.



\subsubsection{Логистическая регрессия}

В данном подразделе быть представлены параметры модели, матрицы ошибок и оценки классификатора для задействованных методов векторизации.



\subsubsection{Перцептрон}

В данном подразделе представлены параметры модели, матрицы ошибок и оценки классификатора для задействованных методов векторизации.

\subsection*{Вывод}

В данном подразделе представлен анализ результатов, сравнение матриц ошибок, сравнительная таблица средних значений оценок классификаторов. Определяется, какой метод имеет самые высокие оценки.
