\phantomsection
\section*{РЕФЕРАТ}
\addcontentsline{toc}{section}{РЕФЕРАТ}

Расчетно-пояснительная записка к выпускной квалификационной работе ``Метод распознавания паттернов суицидального поведения человека по текстовым сообщениям'' содержит \pageref{LastPage} с., \totalfigures\ рис., \totaltables\ табл., 50 ист., 0 прил.

Ключевые слова: суицидология, суицидальность, анализ текста, поведение человека, машинное обучение, классификация.

Объектом разработки является метод распознавания паттернов суицидального поведения человека.

Цель работы: разработка и программная реализация метода распознавания паттернов суицидального поведения человека по текстовым сообщениям.

В аналитическом разделе рассматриваются термины предметной области, представлен интегративный теоретический подход в суицидологии и приведена информация об определении истинности суицидальных намерений. 
Описываются факторы повышенного суицидального риска и мероприятия по предотвращению самоубийств. 
Приводится статистика совершения самоубийств.
С использованием классификации признаков паттернов суицидального поведения человека описываются форматы хранения проявления поведения человека.
Представлена информация о задействованных в настоящее время алгоритмах в задачах классификации сообщений в сети Интернет.
Приводится формализация задачи метода распознавания суицидальных паттернов поведения человека по текстовым сообщениям.

В конструкторском разделе описывается метод распознавания суицидальных паттернов поведения человека по текстовым сообщениям, а также формат и метод сбора задействованных в нем данных.
Рассматриваются средства реализации автоматизированного средства сбора данных.
Приводится диаграмма вариантов использования, декомпозиция задачи распознавания суицидального сообщения, а также диаграмма ``сущность-связь'' в нотации Чена.
Определяется перечень задействованных методов машинного обучения и векторизации, а также приводятся их схемы работы.

В технологическом разделе определяются инструменты разработки средства сбора данных и средства распознавания суицидальных паттернов поведения человека по текстовым сообщениям.
Реализован метод распознавания суицидальных паттернов поведения человека по текстовым сообщениям. 
Представлены интерфейсы разработанных средств. 
Приводится описание обрабатываемых данных, а также анализ тональности сообщений и облаков слов каждого класса.

В исследовательском разделе проведено сравнительное исследование задействованных в методе алгоритмов машинного обучения и определено, какой из рассмотренных методов позволяет достичь лучших метрик точности с использованием матрицы ошибок и графиков оценок классификаторов.

Разработанный метод применим в сфере суицидологии. 
Автоматизированные средства поиска суицидальных сообщений, а также обычные анализаторы сообщений могут задействовать данный метод для обнаружения индивидов в суицидоопасных состояниях, а также для наблюдения людей, пребывающих в периоде пресуицида и постсуицида.

\pagebreak