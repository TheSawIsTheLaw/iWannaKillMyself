\section*{ЗАКЛЮЧЕНИЕ}
\addcontentsline{toc}{section}{ЗАКЛЮЧЕНИЕ}

Были рассмотрены термины предметной области, включающие в себя понятия самоубийства и суицидального поведения. 
Представлен интегративный теоретический подход в суицидологии, описаны суицидальное поведение, внешние проявления суицидального поведения, пресуицид, парасуицид и постсуицид. 
Приведена информация об определении истинности суицидальных намерений. 

Описаны факторы повышенного суицидального риска и мероприятия и методики предотвращения самоубийств, включающие в себя суицидологическую диагностику, кризисную терапию и Телефон Доверия.
Представлена статистика совершения самоубийств.

С использованием классификации признаков паттернов суицидального поведения человека описаны форматы хранения проявления поведения человека. 
Были выделены аудиальные, текстовые, пространственно-временные, визуальные, физиологические и биологические признаки.

Был описан метод распознавания суицидальных паттернов поведения человека по текстовым сообщениям, а также формат и метод сбора задействованных в нем данных. 
В качестве средства сбора данных использован бот в мессенджере Telegram. Рассмотрены доступные средства реализации ботов в выбранном мессенджере.

Был определен перечень задействованных методов машинного обучения, который включил в себя: градиентный бустинг, метод случайного леса, метод опорных векторов, метод K-ближайших соседей, логистическую регрессия и перцептрон. В качестве методов векторизации выбраны: алгоритм ``мешок слов'' и языковая модель BERT.

Была приведена диаграмма вариантов использования. Для системы было определено три действующих лица: пользователь, рекомендатор и анализатор. 
Приведена IDEF0 диаграмма, декомпозирована главная задача метода -- распознавание суицидального сообщения. 
Диаграмма ``сущность-связь'' в нотации Чена позволила на абстрактном уровне описать систему распознавания.

Был разработан метод распознавания суицидальных паттернов поведения человека по текстовым сообщениям, который включил в себя хранение и анализ сообщений пользователей.
Для определения, является ли сообщение суицидальным, используется модель машинного обучения. В качестве обучающей выборки используется дополненный датасет размеченных несуицидальных сообщений из открытого доступа.

Разработанный метод был реализован. Представлены интерфейсы средства сбора данных и средства распознавания суицидальных паттернов поведения человека по текстовым сообщениям.

Представленные диаграммы тональности сообщений показали, что практически треть суицидальных сообщений автоматизированное средство оценки тональности распознает как сообщения с отрицательной окраской. 
Среди несуицидальных сообщений преобладают тексты с отрицательной окраской, при этом нейтральных сообщений -- четверть из всех.

Визуализированные облака слов подтвердили гипотезу, что выбранные классы суицидальных и несуицидальных сообщений разделимы и отличны частотой некоторых слов. 
Отмечено, что слово ``хотеть'' встречается в суицидальных сообщениях в $\approx 7.83$ раза чаще, чем в несуицидальных, а слово ``человек'' -- в $\approx 7.33$ раза чаще. 
Таким образом, суицидальные сообщения являются менее ``разнообразными'' и фиксирующимися на определенном словарном множестве.

Было проведено исследование, включившее в себя построение матриц ошибок и определение метрик точности, F1-меры и ROC-AUC для каждого рассматриваемого алгоритма машинного обучения.
Для определения гиперпараметров каждой модели применялся метод поиска по сетке с опорой на значение F1-меры. Разбиение данных на выборки производилось на 4 части, 1 из которых используется в качестве тестовой.
Лучшее среднее значение всех метрик показал метод случайного леса с использованием BERT-векторизации.
Его точность достигла показателя $0.888$, F1-мера -- $0.887$, а ROC-AUC -- $0.948$.
При этом на втором месте располагается тот же метод, но с использованием векторизации ``Мешок слов'', относительно первого метода его точность уступила на $\approx 1.1\%$, F1-мера -- на $\approx 1.4\%$, а ROC-AUC -- на $\approx 0.1\%$. 
На третьем месте располагается логистическая регрессия с использованием BERT-векторизации, его точность ниже на $\approx 1.6\%$, F1-мера -- на $\approx 2.1\%$, а ROC-AUC -- на $\approx 0.6\%$. 
Таким образом, в качестве используемой модели в задаче распознавания суицидального поведения человека по текстовым сообщениям рекомендуется использовать метод случайного леса с указанными в исследовании параметрами.
Стоит отметить, что данный метод также на $\approx 67\%$ чаще ошибочно интерпретирует обычные сообщения как суицидальные, чем суицидальные как обычные. Данный факт не относится к проблеме модели, которая может помешать работе системы в силу того, что распознавание суицидальных сообщений для нее играет первостепенную роль.

\pagebreak