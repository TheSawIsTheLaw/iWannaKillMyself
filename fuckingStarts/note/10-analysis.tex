\section{Основная часть}

\subsection{Понятие самоубийства и суицидального поведения}
Самоубийство является актом, присущим исключительно человеку. Люди, которые решаются на этот шаг, обычно испытывают глубокую душевную боль и находятся в состоянии сильного стресса, ощущая беспомощность перед своими проблемами. \cite{alimobSuicide}

Среди суицидов можно выделить три основные группы, выявленные учеными: истинный суицид, когда человек действительно желает смерти; демонстративный суицид, который не связан с истинным желанием умереть, а является способом привлечь внимание к своим проблемам; скрытый суицид, когда человек совершает действия, которые с большой вероятностью могут привести к смертельному исходу. \cite{alimobSuicide}

В группы индивидов, подверженных опасности суицида, входят люди с проблемами в области межличностных отношений, а также одинокие и злоупотребляющие наркотиками и алкоголем, сверхкритичные к себе, страдающие от трагических утрат, подвергающиеся насилию и страдающие от болезней. Также следует обращать внимание и на наследственные психологические расстройства.

Суицидальное поведение -- это поведение, которое представляет собой серию реакций, наблюдающихся у человека, когда в его сознании возникает представление о том состоянии дел, которое рассматривается как цель. Существует три типа суицидального поведения \cite{Kasyanov}:
\begin{itemize}
\item самоубийство как решение проблемы (например, "если я умру, то больше не будет никаких проблем");
\item самоубийство как цель, желание (например, "я хочу сдохнуть");
\item саморазрушительное поведение (например, нанесение себе увечий).
\end{itemize}

Суицидальное поведение может интепретироваться как просьба о помощи в 90\% случаев, и лишь в 10\% случаев -- как истинное желание покончить с собой. \cite{Kasyanov}

\subsection{Суицидальное поведение в сети интернет}
В процессе развития суицидальных симптомов человек не перестает искать способы облегчения своих страданий, прежде чем перейти к радикальному решению. Часто, чтобы облегчить негативное душевное состояние, достаточно бывает выговориться, поделиться наболевшим и социальные сети выступают здесь площадкой, где это можно сделать.

Таким образом, социальные сети могут быть использованы в целях выявления суицидентов или людей, склонных к суициадальному поведению.

\subsection{Цель и задачи ВКР}
Цель -- разработать и реализовать метод распознавания паттернов суицидального поведения человека по текстовым сообщениям.

Задачи:
\begin{itemize}
	\item проанализировать предметную область и особенности выявления суицидентов;
	\item определить способ предоставления сервиса и его назначение;
	\item разработать метод распознавания суицидальных паттернов поведения человека по текстовым сообщениям;
	\item реализовать разработанный метод;
	\item дать рекомендации о применимости реализованного метода.
\end{itemize}

\subsection{Обработка естественного языка}

Обработка естественного языка -- это пересечение областей машинного обучения и математической лингвистики, направленное на изучение методов анализа и синтеза естественного языка.

Среди задач обработки естественного языка выделяют:
\begin{itemize}
\item распознавание текста, речи, синтез речи;
\item морфологический анализ;
\item синтаксический разбор и токенизация предложений;
\item извлечение отношений, определение языка, анализ эмоциональной окраски;
\item аннотация документа, перевод, анализ тематики;
\item дедубликация, информационный поиск.
\end{itemize}

\subsection{Задача классификации сообщений}
Задача классификации подразумевает наличие множества объектов, разделенных некоторым образом на классы. В случае решения задачи классификации суицидальных сообщений может быть использовано следующее разделение \cite{presuicidalSignals}:

\begin{itemize}
\item исторические или текущие негативные события - сообщения, носящие фактический характер, описывающие негативные моменты, которые могут произойти с человеком;
\item текущее негативное эмоциональное состояние -- сообщения, содержащие отображение субъективного негативного отношения к себе и окружающим;
\item сообщения о намерении суицида -- сообщения, содержащие декларацию действий, например "завтра в 7 я пойду прыгать с крыши";
\item суицидальная тематика -- сообщения, содержащие суицидальную тематику, но не подпадающие под другие категории;
\item сообщения, не имеющие отношения к суицидальной тематике.
\end{itemize}

Для векторизованного представления данных может быть использована модель BERT \cite{bert} или алгоритмы one-hot encoding и word2vec \cite{word2vec}. Для построения предиктивной модели могут быть использованы методы машинного обучения: градиентный бустинг, метод случайного леса, метод опорных векторов.

\subsection{BERT}

BERT (Bidirectional Encoder Representations from Transformers) - это модель глубокого обучения, разработанная компанией Google, которая представляет собой метод предварительного обучения представлений языка. Она позволяет создавать модели, способные понимать естественный язык, и применять их к различным задачам обработки текстов. BERT обучается на большом корпусе текста, например, на содержимом Википедии, и позволяет создавать модели, которые понимают контекст и смысл слов в предложении. \cite{bert}

BERT является первой ненадзорной системой глубокого бидирекционального предварительного обучения для обработки естественного языка, что означает, что он учитывает контекст с обеих сторон при обработке текста. Благодаря этому BERT может успешно решать задачи, связанные с пониманием естественного языка, такие как вопросно-ответная система, классификация текстов и машинный перевод. BERT является одной из самых успешных моделей для обработки естественного языка и широко применяется в сфере машинного обучения.

Также BERT позволяет с использованием его скрытых слоев получать векторное представление анализируемых предложений, то есть получать вектор коэффициентов для анализа тектовой информации.

Главный недостаток модели заключается в ее сложности, требующей больших вычислительных ресурсов для обучения и использования.

\subsection{One-hot encoding}
One-hot encoding -- это метод преобразования категориальных переменных в числовые признаки, при котором каждая уникальная категория превращается в новый бинарный признак. Каждый бинарный признак соответствует одной из категорий и принимает значение 1, если объект принадлежит этой категории, и 0 в противном случае. Этот метод позволяет сохранить информацию о наличии или отсутствии конкретной категории, при этом избегая установки ненужных порядковых или числовых связей между категориями.

Главная проблема данного метода заключается в сложности представления с его помощью больших данных.

\subsection{word2vec}
word2vec -- способ построения сжатого пространства векторов слов, использующий нейронные сети. Принимает на вход большой текстовый корпус и сопоставляет каждому слову вектор. Сначала он создает словарь, а затем вычисляет векторное представление слов. Векторное представление основывается на контекстной близости: слова, встречающиеся в тексте рядом с одинаковыми словами (а следовательно, имеющие схожий смысл), в векторном представлении имеют высокое косинусное сходство. \cite{word2vec}

word2vec работает на основе двух моделей: Continuous Bag of Words (CBOW) и Skip-gram, причем модель Skip-gram использует целевое слово для предсказания контекста, а в CBOW, наоборот, по контексту подбирается наиболее подходящее слово.

Одним из главных недостатков метода является то, что с его помощью не могут быть представлены слова, которые отсутствуют в обучающей выборке.

\subsection{Классификаторы}

\subsubsection{Градиентный бустинг}
Градиентный бустинг -- это техника машинного обучения для задач классификации и регрессии, которая строит модель предсказания в форме ансамбля слабых предсказывающих моделей, обычно деревьев решений. Обучение ансамбля проводится последовательно. На каждой итерации вычисляются отклонения предсказаний уже обученного ансамбля на обучающей выборке. Следующая модель, которая будет добавлена в ансамбль будет предсказывать эти отклонения. Таким образом, добавив предсказания нового дерева к предсказаниям обученного ансамбля можно уменьшить среднее отклонение модели, которое является целью оптимизационной задачи. Новые деревья добавляются в ансамбль до тех пор, пока ошибка уменьшается, либо пока не выполняется одно из правил ``ранней остановки''. \cite{boosting}

Основные параметры модели:
\begin{itemize}
\item число деревьев;
\item размер шага -- предотвращает переобучение;
\item минимальное изменение значения функции потерь для разделения листа на поддеревья;
\item максимальная глубина дерева;
\item регуляризационные параметры.
\end{itemize}

\subsubsection{Метод опорных векторов}
Метод опорных векторов -- один из наиболее популярных методов обучения, применяемый для решения задач классификации и регрессии. Основная идея метода заключается в построении гиперплоскости, разделяющей объекты выборки оптимальным способом. Алгоритм работает в предположении, что чем больше расстояние между разделяющей гиперплоскостью и объектами разделяемых классов, тем меньше будет средняя ошибка классификатора. \cite{svm}

Основные параметры модели:
\begin{itemize}
\item регуляризационный параметр;
\item тип ядра;
\item степен полиномиальной функции ядра в случае полиномиального;
\item независимый член функции ядра;
\item задействование эвристики сжатия.
\end{itemize}

\subsubsection{Метод случайного леса}
Дерево решений -- это логический алгоритм классификации, решающий задачи классификации и регрессии. Представляет собой оъединение логических условий в структуру дерева. Случайный лес -- это бэггинг над решающими деревьями, при обучении которых для каждого разбиения признаки выбираются из некоторого случайного подмножества признаков. \cite{randomforest}

Основные параметры модели:
\begin{itemize}
\item число деревьев;
\item функция измерения качества изменения;
\item максимальная глубина дерева;
\item минимальное количество выборок, необходимое для разделения внутреннего узла.
\end{itemize}

\subsection{Средство реализации}

При анализе и предобработке датасета, а также для обучения моделей и визуализации данных будет использован язык программирования Python \cite{Python}.

Данный выбор обусловлен следующими факторами:

\begin{itemize}
\item большое количество документации;
\item широкий выбор библиотек для разработки в области машинного обучения;
\item простота синтаксиса языка и высокая скорость разработки.
\end{itemize}

\subsection{Библиотеки}

Для анализа и визуализации данных могут быть задействованы следующие библиотеки:

\begin{itemize}
\item pandas;
\item numpy;
\item matplotlib;
\item scikit-learn;
\item pyTorch;
\end{itemize}

Pandas -- библиотека на языке Python для обработки и анализа данных \cite{pandas}.

Numpy -- это расширение языка Python, добавляющее поддержку больших многомерных массивов и матриц, вместе с большой библиотекой высокоуровневых математических функций для операций с этими массивами \cite{numpy}.

Matplotlib -- библиотека на языке Python для визуализации данных \cite{matplotlib}.

Scikit-learn -- библиотека, позволяющая выполнять множество операций и алгоритмов, используемых в Data Science и Machine Learning \cite{sklearn}. Данная библиотека поддерживает предварительную обработку данных, уменьшение размерности, выбор модели, регрессии, классификации, а также кластерный анализ.

PyTorch -- фреймворк машшиного обучения для языка Python с открытым исходным кодом, созданный на базе Torch \cite{torch}. Используется для решения задач компьютерного зрения и обработки естественного языка.


\pagebreak