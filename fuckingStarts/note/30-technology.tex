\section{Технологический раздел}

В данном разделе описываются средства разработки программного обеспечения и требования к нему. Приводится 

\subsection{Выбор языка и фреймворка для разработки системы}

Для разработки "Распределенной системы распознавания несуицидальных сообщений" должны быть использованы фреймворки, основанные на парадигме MVC:
\begin{itemize}
\item модель представляет объекты, хранимые в системе;
\item вид отвечает за визуализацию моделей;
\item элемент управления отвечает за взаимодействие пользователя и проммного обеспечения.
\end{itemize}

Spring Boot позволяет избавиться от трудоемкой первоначальной установки и настройки среды равертывания. основные преимущества \cite{spring}:

\begin{itemize}
\item быстрая и легкая разработка приложений;
\item автоконфигурация всех компонентов;
\item готовые встроенные серверы, обеспечивающие ускоренное и более продуктивное развертывание приложений;
\item отсутствие конфигурации XML;
\item большой выбор плагинов, облегчающих работу со встроенными базами данных, легкий доступ к ним и служюам очередей;
\item подробная документация.
\end{itemize}

К недостаткам фреймворка относятся:
\begin{itemize}
\item создает множество неиспользуемых зависимостей, что приводит к большому размеру файла развертывания;
\item не подходит для создания монолитных приложений.
\end{itemize}

Фреймворк Spring Boot доступен как для языка программирования Java, так и для языка программирования Kotlin. Для реализации клиентского приложения будет использоваться язык программирования Kotlin, так как в 2019 году компания Google объявила данный язык приоритетным при разработке Android-приложений. В связи с этим, при реализации серверной части приложения, будет использоваться так же язык программирования Kotlin \cite{Kotlin}.

Таким образом, в результате проведенного анализа в качестве языка программирования выбран язык программирования Kotlin, фреймворк -- Spring Boot.

\subsection{Выбор СУБД}
PostgreSQL – реляционная система управления базами данных. Она является некоммерческим ПО с открытым исходным кодом. Для работы с этой СУБД существуют библиотеки для таких распространенных языков программирования, как Python, Ruby, Perl, PHP, C, C++, Java, C\#, Go. Она работает под управлением многих операционных систем: Linux, MacOS, Windows, FreeBSD, Solaris и др. По сравнению с MySQL система PostgreSQL лучше работает с репликацией, так как в ней существует журнал (средство восстановления системы в случае сбоя) физической модификации страниц. PostgreSQL осуществляет асинхронную репликацию типа «ведущий — ведомый». \cite{postgres}

Выбор СУБД PostgreSQL для хранения данных разрабатываемой системы обеспечит надежность, безопасность и масштабируемость.

\subsection{Обеспечение масштабируемости}

СОА позволяет масштабировать систему горизонтально с использованием сервсов-балансировщиков. Совместное использование СОА и REST-стиля предоставляет возможность с легкостью добавлять и удалять сервисы, динамически распределяя нагрузку между существующими. Также имеется возможность воспользоваться глобальным сервисом-координатором.

\subsection{Сборка и деплой системы}

Для автоматической сборки проекта используется функциональность Github Actions \cite{ghactions}. При отправке изменений в master-ветку репозитория автоматически запускается сборка каждого из микросервисов.

Для развертывания системы используется docker compose \cite{dockercompose}, каждый сервис и база данных каждого сервиса разворачиваются в отдельном контейнере. Все контейнеры-микросервисы связаны общей сетью, каждый микросервис связан отдельной сетью с контейнером с его базой данных.

Создание apk-файла Android-приложения не автоматизировано. Для  создания исполняемого файла используется следующая конфигурация build-файла Gradle \cite{gradle}:

\begin{code}
	\captionof{listing}{Build-файл Gradle.}
	\label{code:build}
	\inputminted
	[
	frame=single,
	framerule=0.5pt,
	framesep=20pt,
	fontsize=\small,
	tabsize=4,
	linenos,
	numbersep=5pt,
	xleftmargin=10pt,
	breaklines=true
	]
	{text}
	{code/build.gradle}
\end{code}