\phantomsection
\section*{РЕФЕРАТ}
\addcontentsline{toc}{section}{РЕФЕРАТ}

Расчетно-пояснительная записка к выпускной квалификационной работе ``Метод распознавания паттернов суицидального поведения человека по текстовым сообщениям'' содержит \pageref{LastPage} с., \totalfigures\ рис., \totaltables\ табл., x ист., x прил.

Ключевые слова: суицидология, суицидальность, анализ текста, поведение человека, машинное обучение, классификация.

Объектом разработки является метод распознавания паттернов суицидального поведения человека.

Цель работы: разработка и программная реализация метода распознавания паттернов суицидального поведения человека по текстовым сообщениям.

В аналитическом разделе работы описан интегративный подход в описании суицидального поведения человека. Представлены определения суицида, постсуицида и пресуицида, приведено их значение в контексте решаемой задачи. Описаны мотивы и поводы проявления суицидального поведения, приведена информация о современных методиках предотвращения самоубийств, а также статистика их совершений. Проведен анализ действий и характеристик человека, позволяющих распознать паттерны суицидального поведения. Классифицированы признаки паттернов суицидального поведения человека. Описаны форматы хранения проявления поведения человека.

В конструкторском разделе работы разработан метод распознавания суицидальных паттернов поведения и представлено его описание в виде детализированных диаграмм IDEF0 и диаграммы вариантов использования. Определен перечень включаемых в него характеристик, получаемых от пользователя. Описан способ получения набора данных суицидального поведения. Описан и обоснован перечень методов машинного обучения, задействованных в методе, а также рассматриваемые методы векторизации текста.

В технологическом разделе определены средства реализации программного обеспечения, обоснован выбор используемой СУБД для решения поставленной задачи. Реализован метод распознавания суицидальных паттерном поведения человека по текстовым сообщениям. Представлена визуализация обрабатываемых данных в виде облака слов для каждого класса.

В исследовательском разделе проведено сравнительное исследование задействованных в методе алгоритмов машинного обучения и определено, какой из рассмотренных методов позволяет достичь лучшей точности с использованием матрицы ошибок и графиков оценок классификаторов.

Разработанный метод применим в сфере суицидологии. Автоматизированные средства поиска суицидальных сообщений, а также обычные анализаторы сообщений могут задействовать данный метод для обнаружения индивидов в суицидоопасных состояниях, а также для наблюдения людей, пребывающих в периоде пресуицида и постсуицида.

\pagebreak